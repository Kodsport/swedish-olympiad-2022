\problemname{Short Vowels}
\noindent
Solving algorithm problems is difficult, but one thing that is often even more difficult is preparing the test data. 
Take the problem \textit{Arabic} for example. Here, the jury has spent many hours of intensive work constructing masterpieces such as \texttt{hej vad heter du} (hello what is your name).

A question that arises is: how do you create text strings that do not contain any short vowels? 
If you read the problem \textit{Arabic}, you might remember that a short vowel is a vowel followed by at least two consonants. 
In the word \texttt{tall} (pine tree), the 'a' is a short vowel, while the word \texttt{potatis} (potato) does not have any short vowels. 
For simplicity, we count \textit{a, e, i, o, u, y} as vowels in this problem.

One way to create words that do not contain any short vowels is to start with a word and then remove some letters from it. If we start from \texttt{potatis}, we could get \texttt{ptais} for example. 
But if the word instead became \texttt{otats}, unfortunately, a short vowel would arise.

Your task is to count the number of ways to remove letters from a given word so that the result does not contain any short vowels. It is allowed to not remove any letters at all (in the second example, this contributes $1$ to the answer). 
However, it is not allowed to remove all the letters. If the same word arises by removing different amounts of letters, they are counted separately (In the first example, there are two ways to get the word \texttt{tal}, we can remove either the first or the second \texttt{l}).

\section*{Input}
The input consists of a single line with a word $S$ of at most 50 letters. The word consists only of the letters \texttt{a-z}.

\section*{Output}
Output an integer, the number of ways to remove letters so that a word without short vowels is formed.

Note that the answer will not always fit in a $32$-bit integer in the later test cases.

\section*{Scoring}
Your solution will be tested on a set of test groups, each worth a number of points. Each test group contains
a set of test cases. To get the points for a test group you need to solve all test cases in the test group.

\noindent
\begin{tabular}{| l | l | p{12cm} |}
  \hline
  \textbf{Group} & \textbf{Points} & \textbf{Constraints} \\ \hline
  $1$    & $20$      & All characters in $S$ are the same. \\ \hline
  $2$    & $40$      & $|S| \leq 10$ \\ \hline
  $3$    & $40$      & No additional constraints. \\ \hline
\end{tabular}
