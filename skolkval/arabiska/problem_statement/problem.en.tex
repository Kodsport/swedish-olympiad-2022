\problemname{Arabic}
\noindent
It is quite well known that when writing text in Arabic, one writes from right to left.
However, it is less well known that short vowels are never written out.

By vowels, we mean one of the letters: \textit{a, e, i, o, u, y}.
A short vowel is defined in this problem as a vowel that is followed by two or more consonants.

Write a program that shows how a sentence would look if it were written in Arabic, i.e., from right to left and omitting all short vowels from the original sentence.

\section*{Input}
The first line contains an integer $N$ ($1 \le N \le 5$), the number of words in the sentence.

The second line contains $N$ words, where each word consists of at most $10$ lowercase letters from the Latin alphabet (\texttt{a} to \texttt{z}).

\section*{Output}
Output $N$ words: how the sentence would be written in Arabic.

\section*{Scoring}
Your solution will be tested on a set of test groups, each worth a number of points. Each test group contains
a set of test cases. To get the points for a test group you need to solve all test cases in the test group.

\noindent
\begin{tabular}{| l | l | p{12cm} |}
  \hline
  \textbf{Group} & \textbf{Points} & \textbf{Constraints} \\ \hline
  $1$   & $40$        & There are no short vowels in the sentence. \\ \hline
  $2$   & $60$        & No additional constraints. \\ \hline
\end{tabular}