\problemname{Grönt Kort}


För att repklättra krävs två personer, en som klättrar och en som står kvar på marken och håller i repet (säkrar) utifall att klättraren skulle falla ner. För att få säkra krävs att man tagit grönt kort. Däremot behöver man inte ha grönt kort för att få klättra. Att klättra en vägg, inklusive att knyta fast repet i selen och allt runtomkring, tar exakt 10 minuter. Det finns många klätterväggar, så hur många personer som helst kan klättra samtidigt (men de måste bli säkrade av olika personer).

Ett kompisgäng består av $N$ personer med grönt kort och $M$ personer utan grönt kort. Hur många minuter tar det som minst innan alla har fått klättra en gång?

\section*{Indata}
Den första raden inehåller ett heltal $N$ ($2 \le N \le 400\,000\,000$), antalet personer med grönt kort.

Den andra raden inehåller ett heltal $M$ ($0 \le M \le 400\,000\,000$), antalet personer utan grönt kort.


\section*{Utdata}
Skriv ut ett heltal -- det minsta antalet minuter innan alla $N+M$ personerna har fått klättra.

\section*{Poängsättning}
För 1 poäng gäller att $M = 0$. \\
För ytterligare 1 poäng gäller att $N = 2$.

\section*{Förklaring av exempel}
I det första exempelfallet så finns det två personer, båda med grönt kort.
Den ena personen säkrar när den andra klättrar, och sen kan de byta vem som klättrar och säkrar. Totalt tar det 20 minuter innan båda personerna har fått klättra.

I det andra exempelfallet så finns det fyra personer, varav två har grönt kort.
De två personerna med grönt kort kan båda få klättra under de första 20 minuterna (som i
det första expempelfallet). Sedan kan de två personerna utan grönt kort klättra samtidigt.
Totalt tar det alltså 30 minuter innan alla personerna har fått klättra.

I det tredje exempelfallet så finns det sex personer, varav tre har grönt kort.
Ett sätt för dem att klättra på 30 minuter är att alltid två personer klättrar samtidigt, en med grönt kort och en utan grönt kort.

\section*{Exempelfall}

