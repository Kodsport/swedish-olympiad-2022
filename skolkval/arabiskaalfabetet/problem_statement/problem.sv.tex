\problemname{Det arabiska alfabetet}
Det är ganska känt att när man skriver text på arabiska skriver man från höger till vänster.
Det är däremot mindre välkänt att man aldrig skriver ut korta vokaler.

Med vokaler menar vi en av bokstäverna: \textit{a,e,i,o,u,y}.
En kort vokal defineras i detta problem som en vokal som följs av två eller fler konsonanter.

Du kommer få en mening som använder bokstäverna \texttt{a} till \texttt{z} samt vissa specialtecken.
Skriv ett program som visar hur meningen skulle se ut om den skrevs som om det vore på arabiska, d.v.s från höger till vänster och utan några av dess korta vokaler).

\section*{Indata}
Den första raden inehåller ett heltal $N$ ($1 \le N \le 10$), antalet ord i meningen.

Den andra raden innehåller $N$ ord, där varje ord består av högst 30 tecken. Alla tecken är antingen en bokstav \texttt{a} till \texttt{z} eller något av specialtecken: ``\texttt{.,?!\_}'' (citationstecknen är inte inkluderade).

\section*{Utdata}
Skriv utt en mening bestående av $N$ ord -- hur meningen skulle skrivas på arabiska.

\section*{Poängsättning}
För 2 poäng gäller att det inte finns några korta vokaler. \\
