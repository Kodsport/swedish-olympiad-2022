\problemname{Ljusshow}
Din vän håller på att designa en ljusshow för avslutningsceremonin i årets Programmeringsolympiadsfinal.
Salen där ceremonin hålls kan ses som ett rutnät med $R$ rader och $C$ kolumner.
Utmed de fyra sidorna är olika lampor monterade, vilka kan lysa med en av tre olika färger: rött, blått eller grönt.
Under ceremonin är tanken att lamporna skiftar i olika mönster.

En lampa lyser upp samtliga rutor längs med samma kolumn eller rad som den är monterad längs.
Om en viss ruta lyses upp av minst en lampa av varje färg kommer ljuset i rutan att uppfattas som ett otrevligt bländande vitt.
Din vän har redan designad ett utkast till ljusshowen, men undrar nu om vissa av de ljuskonfigurationer han valt orsakar att för många rutor blir vita.
För att kunna avgöra om en konfiguration är okej eller inte har du fått i uppgift att skriva ett program som läser in vilken färg samtliga lampor ska lysa med, och beräknar antalet rutor som kommer lysa vitt.

\section*{Indata}
\begin{itemize}
  \item
    Den första raden innehåller två heltal: $R$ ($1 \le R \le 10^6$) och $C$ ($1 \le C \le 10^6$), antalet rader och kolumner i den rutnäts-formad salen.

  \item
    De fyra nästa raderna innehåller en textsträng vardera som beskriver färgerna beskriver vilka färger alla lampor har.
    Den första raden beskriver de $C$ lamporna i toppen av rutnätet som skiner nedåt i ordning vänster till höger,
        den andra de $R$ lamporna i rutnätet till höger om rutnätet som skiner till vänster i ordning uppifrån och ned,
        den tredje de $C$ lamporna i rutnätet under rutnätet som skiner uppåt i ordning vänster till höger,
        den fjärde de $R$ lamporna i rutnätet till vänster om rutnätet som till höger i ordning uppifrån och ned.

    Färgen på en lampa beskrivs med hjälp av tecknen \texttt{RGB} beroende på om lampan lyser rött, grönt eller blått.
\end{itemize}

\section*{Utdata}
Skriv ut ett heltal -- antalet rutor i salen som lyser vitt.

\section*{Poängsättning}
Din lösning kommer att testas på en mängd testfallsgrupper.
För att få poäng för en grupp så måste du klara alla testfall i gruppen.

\noindent
\begin{tabular}{| l | l | l |}
  \hline
  Grupp & Poängvärde & Gränser \\ \hline
  $1$    & $20$        &  Gruppen består av ett enda testfall, det som finns på vår affisch (\url{https://www.progolymp.se/2022/affisch.pdf}). \\ \hline 
  $2$    & $10$        &  Alla lampor på samma sida har samma färg. \\ \hline
  $3$    & $20$        &  $R,C \le 1000$ \\ \hline 
  $4$    & $25$        &  Alla lampor till höger eller vänster om rutnätet lyser rött, och alla lampor över och under rutnätet lyser grönt eller blått. \\ \hline
  $5$    & $15$        &  Inga ytterligare begränsningar. \\ \hline
\end{tabular}
