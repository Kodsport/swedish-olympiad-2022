\problemname{Ljusshow 2}

Efter att du löst problemet \textit{Ljusshow} så inser du att det vita bländande ljuset kan
utnyttjas för att lysa upp vissa av rutorna i rutnätet. I det här problemet får du givet vilka av
rutorna som ska lysa vitt och vilka som inte ska göra det, och din uppgift är att placera ut lamporna
längs kanten så att så många av kraven som möjligt blir uppfyllda.


\section*{Indata}
Indatan består av $10$ testfall.

\begin{itemize}
  \item 
    Den första raden innehåller ett heltal $T$ ($0 \leq T \leq 10$), numret på testfallet ($0$ är exempelfallet nedan).
  \item
    Den andra raden innehåller två heltal: $n$ och $m$ ($1 \le n,m \le 1000$), antalet rader och kolumner i rutnätet.
  \item

    De följande $n$ raderna utgör en beskrivning av vilka rutor som ska lysa i rutnätet och vilka som inte ska det.
    Varje rad kommer bestå av en sträng med $m$ ettor och nollor. En etta på rad $r$ och kolumn $c$ indikerar att den
    rutan ska lysa vitt. En nolla indikerar att rutan inte ska lysa vitt.

\end{itemize}

\section*{Utdata}
Skriv ut fyra rader med en sträng på varje. Strängarna ska utgöra en utplacering av lampor, på samma format
som indatan i \textit{Ljusshow}. Notera att du inte ska skriva ut $n$ och $m$.

\section*{Poängsättning}
Ditt program kommer att testas på $10$ testfall, och poängsättas baserat på hur det står sig relativt domarnas lösning.

En lösnings poäng på ett specifikt testfall definieras som $\max(0, x - \frac{nm}{2})$, där $x$ är 
antalet rutor i rutnätet där ljuset från lamporna som lösningen placerade ut uppfyller de givna kraven
på vilka rutor som skulle lysa vitt och inte. Om din lösning fick poängen $P_1$ och domarlösningen fick
poängen $P_2$ så får du $10 \cdot (P_1 / P_2)$ poäng på testfallet, avrundat till två decimaler.

Din poäng för en inskickning är summan av poängen för testfallen, och din totala poäng är poängen för den bästa inskickningen.
Det går alltså inte att kombinera testfall mellan olika inskickningar, utan du måste skicka in ett enda program som
maximerar din poäng på alla testfall samtidigt.

Din lösning kan maximalt få $100$ poäng, även om den är bättre än den domarna skrivit.

Exempeltestfallet kommer alltid att ge $0$ poäng.


\section*{Förklaring av exampelfall}
I exempelfallet så uppfyller lösningen alla kraven och dess poäng är därför $7.5$, vilket är det
maximala möjliga.

\section*{Testfall}
Här är beskrivningar av de $10$ testfallen som förekommer.
Testfall av typ 1 har genererats genom att ta resultatet av en slumpmässig utplacering av lampor.
Det finns alltså alltid ett sätt att uppnå det maximala $\frac{nm}{2}$ poäng på dessa testfall.
Testfall av typ 2 har genererats genom att bara slumpa för varje ruta om den ska vara ett eller noll,
med $50\%$ sannolikhet. Dessa testfall har ofta ingen lösning som uppfyller alla kraven.

\noindent
\begin{tabular}{| l | l | l |}
\hline
Testfall & Gränser                                     & Typ  \\ \hline
$1$       & $n = 10$, $m = 10$                           & 1  \\ \hline
$2$       & $n = 40$, $m = 40$                           & 1   \\ \hline
$3$       & $n = 150$, $m = 150$                           & 1   \\ \hline
$4$       & $n = 400$, $m = 400$                          & 1  \\ \hline
$5$       & $n = 1000$, $m = 1000$                          & 1   \\ \hline
$6$       & $n = 100$, $m = 10$                           & 1   \\ \hline
$7$       & $n = 1000$, $m = 1$                           & 1   \\ \hline
$8$       & $n = 10$, $m = 10$                           & 2  \\ \hline
$9$       & $n = 100$, $m = 100$                      & 2   \\ \hline
$10$      & $n = 1000$, $m = 1000$                       & 2   \\ \hline
\end{tabular}
