\problemname{Örnattack}
Ekorrarna och örnarna har krigat sedan urminnes tider. Gammelekorren är en mästare på astrologi och har förutspått att örnarna snart kommer försöka sig på en sista attack mot storträdet.
Enligt gammelekorren kommer $K$ örnar flyga in i trädet i hög fart för att få hela trädet att skaka så att ekorrarna riskerar att ramla ned.
Han har förutspått i vilken av trädets $N$ noder var och en av örnarna kommer krascha och med vilken fart.
Gammelekorren vill veta hur mycket varje nod kommer skaka under attacken för att kunna varna alla ekorrar om de farligaste noderna.
Tyvärr är gammelekorren inte lika bra på programmering som på astrologi, så han har anställt dig för att räkna ut hur mycket varje nod kommer skaka under attacken.

När en örn kraschar med farten $v$ i nod $u$ börjar nod $u$ skaka med styrkan $v$. Skakningen sprider sig sedan genom kanter ut från noden. Vid varje nod delas skakstyrkan upp i lika delar till varje granne (utom den den kom ifrån). 
Vi kan anta att skakningen hinner sprida sig genom hela trädet och försvinna innan nästa örn kraschar. För var och en av noderna i trädet vill gammelekorren veta hur mycket
noden totalt kommer skaka, dvs. summan av alla skakningars styrkor som noden kommer utsättas för.

\section*{Indata}
På den första raden finns heltalet $N$ -- antalet noder i trädet.
De följande $N-1$ raderna innehåller två heltal $a$ och $b$ ($1 \le a,b \le N$), vilket betyder att det går en kant mellan nod $a$ och nod $b$.
Därefter följer en rad med heltalet $K$ -- antalet örnar som kommer attackera.
Slutligen finns $K$ rader som beskriver örnarna. På varje rad finns två heltal, noden $u$ ($1 \le u \le N$) där örnen kommer krascha och örnens fart $v$ ($1 \le v \le 10^9$).

\section*{Utdata}
Skriv ut hur mycket varje nod totalt kommer skaka på var sin rad. På rad $i$ ska den totala skakningen i nod $i$ stå. Varje tal får max ha ett absolut eller relativt fel på $10^{-5}$.

\section*{Poängsättning}
Din lösning kommer att testas på en mängd testfallsgrupper.
För att få poäng för en grupp så måste du klara alla testfall i gruppen.

\noindent
\begin{tabular}{| l | l | l |}
  \hline
  Grupp & Poängvärde & Gränser \\ \hline
  $1$    & $15$       &  Den $i$:te kanten går mellan nod $i$ och $i+1$ (för $1 \le i \le N-1$) \\ \hline 
  $2$    & $30$       &  $1 \le N,K \le 2000$ \\ \hline
  $3$    & $5$        &  $1 \le N \le 2000, 1 \le K \le 100000 $ \\ \hline
  $4$    & $50$       &  Inga ytterligare begränsningar \\ \hline
\end{tabular}

\section*{Förklaring av exempelfall 2}
Första örnen kraschar i nod 1 med fart 10. Därifrån sprids skakningen till nod 2 och 3. Från nod 2 har skakningen ingenstans att sprida sig men från nod 3 sprids den vidare till nod 4 och 5.
Andra örnen kraschar i nod 4 med fart 6. Från nod 4 sprids skakningen endast till nod 3 och från nod 3 till både nod 1 och 5. Från nod 5 har skakningen ingenstans att ta vägen men från nod 1 sprids den till nod 2.
Den tredje och sista örnen kraschar i nod 3 med fart 5. Sedan sprids skakningen till nod 3 och därifrån till nod 4 och 1. Skakningen i nod 4 har ingenstans att ta vägen men skakningen från nod 1 sprids till nod 2.
