\documentclass[a4paper,12pt,oneside]{amsbook}
\usepackage[T1]{fontenc}
\usepackage{textcomp}
\usepackage{multicol}
%\usepackage{euler}
%\usepackage{garamond}
\usepackage{hyperref}

\usepackage{newcent}
%\fontfamily{fxm}\selectfont
\usepackage{fancyhdr}
\pagestyle{fancy}
% Detta paket skoter grafiken
\usepackage[dvips]{graphicx}
%\usepackage[pdftex]{graphicx}

\usepackage{amsthm}
\usepackage{amsmath}
\usepackage[swedish]{babel}
% For bortkommentering
\usepackage{verbatim}
% For att slippa indentering av Sections och SubSections
\usepackage{indentfirst}

% Har stalls paragrafindentering och avstand mellan stycken och rader mm
\setlength{\parindent}{0pt}
\setlength{\parskip}{6pt}
\setlength{\baselineskip}{12pt}
% Har definieras sidan
\setlength{\voffset}{-15mm}
\setlength{\hoffset}{-5.5mm}

\setlength{\topmargin}{0mm}
\setlength{\headheight}{10mm}
\setlength{\headsep}{8mm}
\setlength{\footskip}{16mm}

\setlength{\evensidemargin}{0mm}
\setlength{\oddsidemargin}{0mm}

\setlength{\marginparsep}{0mm}
\setlength{\marginparwidth}{0mm}

\setlength{\textwidth}{170mm}
\setlength{\textheight}{230mm}
\setlength{\paperwidth}{210mm}
\setlength{\paperheight}{297mm}
\setlength{\headwidth}{170mm}
% Huvud och Fot
%\fancypagestyle{}{
\fancyhf{}\fancyhead[c]{\small{\textit{Programmeringsolympiaden Kvalificering 2020}}}
\fancyfoot[c]{\thepage}
\renewcommand{\headrulewidth}{0.4pt}
\renewcommand{\footrulewidth}{0.4pt}
\newenvironment{lista}
{\begin{itemize}
\setlength{\parindent}{0pt}
\setlength{\itemsep}{6pt}}
{\end{itemize}}
% Uppgifter
\newcounter{probnr}
\newenvironment{tal}{%
\begin{list}
%{\textbf{\arabic{section}.\arabic{probnr}}} {\usecounter{probnr}
{\textbf{\arabic{probnr}}} {\usecounter{probnr}
\setlength{\leftmargin}{0mm}
\setlength{\rightmargin}{0mm}
\setlength{\labelwidth}{-1mm}
\setlength{\labelsep}{1mm}}
\setlength{\itemsep}{6pt}
}{\end{list}}
% Definition av avsnitt: FAKTA, EXEMPEL, PASTAENDE
\newtheorem{fakta}{Fakta}
\newtheorem{exempel}{Exempel}
\newtheorem{problem}{Problem}
%
\newtheoremstyle{test}% NAME
{20pt}      % ABOVESPACE
{10pt}      % BELOWSPACE
{\sffamily} % BODYFONT
{0pt}       % INDENT
{\scshape}  % HEADFONT
{}          % HEADPUNCT
{\newline}  % HEADSPACE
{}          % CUSTOM-HEAD-SPEC

\theoremstyle{test}
\newtheorem{program}{Program}
\newcommand{\sv}[1]{\textsc{#1}}            % Sma versaler
\newcommand{\fe}[1]{\textbf{#1}}            % FET
\newcommand{\ku}[1]{\textit{#1}}            % KURSIV
\newcommand{\cu}[1]{\texttt{#1}}            % Courier
\newcommand{\sk}[1]{\texttt{#1}}            % Courier
\newcommand{\rubrik}[1]{\begin{center}\sf\huge{#1}\normalsize\rm\end{center}}
\begin{document}
%\DeclareGraphicsExtensions{.jpg,.pdf,.mps,.png,.eps}

\problemname{Mötet}

%TODO: improve statement, write story...

En styrelse med $N$ medlemmar planerar att ha ett möte. På grund av det stora antalet medlemar i stryelsen är det svårt att ha en tid som passar alla, men man vill gärna att så många personer som möjligt kan vara med i mötet.\\ Personen $i$ säger att $m_i$ tidsintervall passar dem.\\
När en person säger att tidsintervallet $[a,b]$ passar dem så bteyder det att om mötet startar vid någon tid $t$ sdär $a\leq t \leq b$ så kan de vara med.\\
Vad är det maximala antalet personer som kan vara med i mötet?\\
Notera att personerna i styrelsen är slarviga och kan säga att de kan på intervallen $[a_x,b_x]$, $[a_y,b_y]$ , $x \neq y$ även om intervallen $[a_x,b_x]$, $[a_y,b_y]$ har gemensamma punkter. Exempelvis kan en person säga att de kan på tidsintervallet $[1,3]$ och $[2,4]$.
\section*{Indata}
Den första raden inehåller ett heltal $N \leq 2\cdot 10^5$.

Därefter följer $N$ rader. Rad $i$ börjar med ett heltal $1 \leq m_i \leq 2\cdot 10^5 $ som följs av $m_i$ tidsintervall ($2 \cdot m_i$ heltal). Varje intervall $j$ är i formen $a_i_j\; b_i_j$ , $0\leq a_i_j\leq b_i_j<10^9$ där $a_i_j$ och  $b_i_j$ är hetlat 

Låt $B=\sum_{i=1}^{N} m_i$ (summan av alla $m_i$), då gäller det att i alla testfall $B \leq 2\cdot 10^5$

\section*{Utdata}
Skriv ut en rad med ett heltal -- det största antalet personer som kan vara med i ett möte om man väljer en optimal tid att start mötet i

\section*{Poängsättning}
Din lösning kommer att testas på en antal testfallsgrupper.
För att få poäng för en grupp så måste du klara alla testfall i gruppen.

\noindent
\begin{tabular}{| l | l | l |}
  \hline
  Grupp & Poängvärde & Gränser \\ \hline
  $1$   & $10$        & $B \leq 100$ , $0 \leq a_i_j \leq b_i_j  \leq 100$ \\ \hline
  $2$   & $15$       & $B \leq  1000$ , $0 \leq a_i_j \leq b_i_j  \leq 1000$, och personen $i$ har inga två intervall som delar punkter. \\ \hline
  $3$   & $30$       & personen $i$ har inga två intervall som delar punkter. \\ \hline
  $4$   & $15$       & $B \leq 1000$ , $0 \leq a_i_j \leq b_i_j  \leq 1000$ \\ \hline
  $5$   & $30$       & Inga ytterligare begränsningar \\ \hline
\end{tabular}
Notera att vissa exempelfall är inte giltiga i vissa testfallsgrupper.

\section*{Förklaring av exempel 1}
Exempelfall $1$ är ett exmepel på ett testfall som kan vara med i Grupp $2$ och Grupp $3$.\\ Om vi exmpelvis väljer att starta mötet vid tiden $4$ då kan person $2$ och person $3$ vara med.\\ Notera att exmpelfall $2$ och $3$ kan inte förekomma i Grupp $2$ och Grupp $3$
%TODO förklara exempelfall
\end{document}
