\problemname{Mötet}

En styrelse med $N$ medlemmar planerar att ha ett möte. På grund av det stora antalet medlemar i stryelsen är det svårt att ha en tid som passar alla, men man vill gärna att så många personer som möjligt kan vara med i mötet.\\ Personen $i$ säger att $m_i$ tidsintervall passar dem.\\
När en person säger att tidsintervallet $[a,b]$ passar dem så bteyder det att om mötet startar vid någon tid $t$ sdär $a\leq t \leq b$ så kan de vara med.\\
Vad är det maximala antalet personer som kan vara med i mötet?\\
Notera att personerna i styrelsen är slarviga och kan säga att de kan på intervallen $[a_x,b_x]$, $[a_y,b_y]$ , $x \neq y$ även om intervallen $[a_x,b_x]$, $[a_y,b_y]$ har gemensamma punkter. Exempelvis kan en person säga att de kan på tidsintervallet $[1,3]$ och $[2,4]$.
\section*{Indata}
Den första raden inehåller ett heltal $N \leq 2\cdot 10^5$.

Därefter följer $N$ rader. Rad $i$ börjar med ett heltal $1 \leq m_i \leq 2\cdot 10^5 $ som följs av $m_i$ tidsintervall ($2 \cdot m_i$ heltal). Varje intervall $j$ är i formen $a_i_j\; b_i_j$ , $0\leq a_i_j\leq b_i_j<10^9$ där $a_i_j$ och  $b_i_j$ är hetlat 

Låt $B=\sum_{i=1}^{N} m_i$ (summan av alla $m_i$), då gäller det att i alla testfall $B \leq 2\cdot 10^5$

\section*{Utdata}
Skriv ut en rad med ett heltal -- det största antalet personer som kan vara med i ett möte om man väljer en optimal tid att start mötet i

\section*{Poängsättning}
Din lösning kommer att testas på en antal testfallsgrupper.
För att få poäng för en grupp så måste du klara alla testfall i gruppen.

\noindent
\begin{tabular}{| l | l | l |}
  \hline
  Grupp & Poängvärde & Gränser \\ \hline
  $1$   & $10$        & $B \leq 100$ , $0 \leq a_i_j \leq b_i_j  \leq 100$ \\ \hline
  $2$   & $15$       & $B \leq  1000$ , $0 \leq a_i_j \leq b_i_j  \leq 4000$, och personen $i$ har inga två intervall som delar punkter. \\ \hline
  $3$   & $30$       & personen $i$ har inga två intervall som delar punkter. \\ \hline
  $4$   & $15$       & $B \leq 1000$ , $0 \leq a_i_j \leq b_i_j  \leq 4000$ \\ \hline
  $5$   & $30$       & Inga ytterligare begränsningar \\ \hline
\end{tabular}
Notera att vissa exempelfall är inte giltiga i vissa testfallsgrupper.

\section*{Förklaring av exempel 1}
Exempelfall $1$ är ett exmepel på ett testfall som kan vara med i Grupp $2$ och Grupp $3$.\\ Om vi exmpelvis väljer att starta mötet vid tiden $4$ då kan person $2$ och person $3$ vara med.\\ Notera att exmpelfall $2$ och $3$ kan inte förekomma i Grupp $2$ och Grupp $3$
