\documentclass[a4paper,12pt,oneside]{amsbook}
\usepackage[T1]{fontenc}
\usepackage{textcomp}
%\usepackage{euler}
%\usepackage{garamond}

\usepackage{newcent}
%\fontfamily{fxm}\selectfont
\usepackage{fancyhdr}
\pagestyle{fancy}
% Detta paket skoter grafiken
\usepackage[dvips]{graphicx}
%\usepackage[pdftex]{graphicx}
\usepackage{tabularx}
\usepackage{amsthm}
\usepackage{amsmath}
\usepackage[swedish]{babel}
% For bortkommentering
\usepackage{verbatim}
% For att slippa indentering av Sections och SubSections
\usepackage{indentfirst}

% Har stalls paragrafindentering och avstand mellan stycken och rader mm
\setlength{\parindent}{0pt}
\setlength{\parskip}{6pt}
\setlength{\baselineskip}{12pt}
% Har definieras sidan
\setlength{\voffset}{-15mm}
\setlength{\hoffset}{-5.5mm}

\setlength{\topmargin}{0mm}
\setlength{\headheight}{10mm}
\setlength{\headsep}{8mm}
\setlength{\footskip}{16mm}

\setlength{\evensidemargin}{0mm}
\setlength{\oddsidemargin}{0mm}

\setlength{\marginparsep}{0mm}
\setlength{\marginparwidth}{0mm}

\setlength{\textwidth}{170mm}
\setlength{\textheight}{225mm}
\setlength{\paperwidth}{210mm}
\setlength{\paperheight}{297mm}
\setlength{\headwidth}{170mm}
% Huvud och Fot
%\fancypagestyle{}{
\fancyhf{}\fancyhead[c]{\small{\textit{Svar och rättningsanvisningar
      till Programmeringsolympiadens skolkval 2022}}}
\fancyfoot[c]{\thepage}
\renewcommand{\headrulewidth}{0.4pt}
\renewcommand{\footrulewidth}{0.4pt}
\newenvironment{lista}
{\begin{itemize}
\setlength{\parindent}{0pt}
\setlength{\itemsep}{6pt}}
{\end{itemize}}
% Uppgifter
\newcounter{probnr}
\newenvironment{tal}{%
\begin{list}
%{\textbf{\arabic{section}.\arabic{probnr}}} {\usecounter{probnr}
{\textbf{\arabic{probnr}}} {\usecounter{probnr}
\setlength{\leftmargin}{0mm}
\setlength{\rightmargin}{0mm}
\setlength{\labelwidth}{-1mm}
\setlength{\labelsep}{1mm}}
\setlength{\itemsep}{6pt}
}{\end{list}}
% Definition av avsnitt: FAKTA, EXEMPEL, PASTAENDE
\newtheorem{fakta}{Fakta}
\newtheorem{exempel}{Exempel}
\newtheorem{problem}{Problem}
%
\newtheoremstyle{test}% NAME
{20pt}      % ABOVESPACE
{10pt}      % BELOWSPACE
{\sffamily} % BODYFONT
{0pt}       % INDENT
{\scshape}  % HEADFONT
{}          % HEADPUNCT
{\newline}  % HEADSPACE
{}          % CUSTOM-HEAD-SPEC

\theoremstyle{test}
\newtheorem{program}{Program}
\newcommand{\sv}[1]{\textsc{#1}}            % Sma versaler
\newcommand{\fe}[1]{\textbf{#1}}            % FET
\newcommand{\ku}[1]{\textit{#1}}            % KURSIV
\newcommand{\cu}[1]{\texttt{#1}}            % Courier
\newcommand{\sk}[1]{\texttt{#1}}            % Courier
\newcommand{\rubrik}[1]{\begin{center}\sf\huge{#1}\normalsize\rm\end{center}}
\begin{document}
%\DeclareGraphicsExtensions{.jpg,.pdf,.mps,.png,.eps}


\specialsection*{Svar och rättningsanvisningar}
\thispagestyle{fancy}
\lhead{}
\begin{itemize}
%\setlength\itemsep{0.2cm}
\item Läs igenom \textit{tävlingsreglerna}.
\item Programmen tas i tur och ordning in i editorn och kompileras.
Uppstår kompileringsfel betraktas programmet som felaktigt och lösningen
ges $0$ poäng.
\item Programmet körs med givna indata enligt nedan. Alternativt, om eleven förberett programmet för det, kan de bifogade indatafilerna användas istället.
\item Varje testfall
  med korrekt svar ger 1 poäng.
\item
Totalt kan man
  alltså få 5 poäng för varje uppgift.
\item Om exekveringstiden för ett testexempel, kört på en modern dator,
\textit{överskrider 3 sekunder} betraktas körningen av testexemplet som felaktigt.
\item Det kan vara viktigt att programmet körs i en miljö liknande den som
programmet utvecklats i, samma version av kompilator eller
interpretator.
\item Vid problem i samband med rättningen är det viktigt att det sunda
förnuftet får råda!
\item Ett förslag till rättningsprocedur kan vara att låta eleven
sitta vid datorn.
\end{itemize}

\vspace{2cm}

\subsection*{Uppgift 1 -- Affischutskick}
~\\
{\tt 
\begin{tabular}{||l||c|c|c||c||}\hline\hline
& \multicolumn{3}{c||}{\fe{Indata}} & \fe{Utdata} \\ 
& Kuvert & Affisch & Blad & \\ \hline \hline
\fe{Test 1} & 50 & 51 & 52 & 23.386320  \\ \hline
\fe{Test 2} & 55 & 200 & 107 &  64.731150  \\ \hline
\fe{Test 3} & 142 & 142 & 142 &  65.354364 \\ \hline
\fe{Test 4} & 121 & 144 & 169 & 64.421082\\ \hline
\fe{Test 5} & 199 & 178 & 180 & 85.164048  \\ \hline\hline
\end{tabular}
}
%$$

\subsection*{Uppgift 2 -- Arabiska}
~\\
{\tt 
\begin{tabular}{||l||c|l||l||}\hline\hline
& \multicolumn{2}{c||}{\fe{Indata}} & \fe{Utdata} \\ 
& Antal ord & Mening & \\ \hline \hline
\fe{Test 1} & 1 & aaaaa & aaaaa \\ \hline
\fe{Test 2} & 3 & likadan som fore & erof mos nadakil \\ \hline
\fe{Test 3} & 1 & abcdefghij & jihgfdcb\\ \hline
\fe{Test 4} & 3 & roligt att ratta & attr tt tglor \\ \hline
\fe{Test 5} & 2 & heeejj yrsel & lesr jjeeh \\ \hline\hline
\end{tabular}
}
%$$


\subsection*{Uppgift 3 -- Grönt kort}
~\\
{\tt 
\begin{tabular}{||l||c|c||c||}\hline\hline
& \multicolumn{2}{c||}{\fe{Indata}} & \fe{Utdata} \\ 
& N & M & \\ \hline \hline
\fe{Test 1} & 399999999 & 0 & 30 \\ \hline
\fe{Test 2} & 2 & 123123123 & 615615640 \\ \hline
\fe{Test 3} & 12 & 97 & 110 \\ \hline
\fe{Test 4} & 43 & 86 & 40 \\ \hline
\fe{Test 5} & 3 & 91919191 & 306397330  \\ \hline\hline
\end{tabular}
}
%$$


\subsection*{Uppgift 4 -- Den trötte målaren}
~\\
OBS! Eftersom det finns flera lösningar behöver du verifiera att
elevens svar är korrekt genom att klistra in det på följande webbsida: \\
http://www.progolymp.se/XXXXXXXXXXXXXX
~\\
{\tt 
\begin{tabular}{||l||c|l||l||}\hline\hline
& \multicolumn{2}{c||}{\fe{Indata}} & \fe{Utdata} \\
& $N$ & Rader & \\ \hline \hline
\fe{Test 1} &4&VSVV & Flera möjliga \\
&&VVVV&\\
&&VSSS&\\
&&VSVV&\\\hline
\fe{Test 2} &4&SSSV & Flera möjliga \\
&&VVVV&\\
&&SVVV&\\
&&SV.V&\\\hline
\fe{Test 3} &7&VSVVVSV & Flera möjliga \\
&&VS.SSSV&\\
&&VSVSVSV&\\
&&VSSSSSV&\\
&&VSVSVSV&\\
&&VSSSSSS&\\
&&VS.SSSV&\\\hline
\fe{Test 4} &6&SSVSS. & Flera möjliga \\
&&SSVSSS&\\
&&SSVSSV&\\
&&VSVSVV&\\
&&VVVSVV&\\
&&VVVVVV&\\\hline
\fe{Test 5} &8&VVVVVVSV & Flera möjliga \\
&&SSSSSVSS&\\
&&SSSSSVSV&\\
&&VVVVSVSV&\\
&&VVVVVVVV&\\
&&SVVSSVSV&\\
&&SVVVSVSV&\\
&&SVSSSVSV&\\\hline\hline
\end{tabular}
}


\subsection*{Uppgift 5 -- Korta vokaler}
~\\
{\tt 
\begin{tabular}{||l||l||c||}\hline\hline
& {\fe{Indata}} & \fe{Utdata} \\ \hline \hline
\fe{Test 1} & aaaaaaaaaa & 1023\\ \hline
\fe{Test 2} & algoritm & 143 \\ \hline
\fe{Test 3} & jogurtkaka & 527\\ \hline
\fe{Test 4} & skolkvaletprogrammeringsolympiaden & 413109887\\ \hline
\fe{Test 5} & exempelmeningenidettatestfallharprecisfemtiotecken & 808841612031\\ \hline\hline
\end{tabular}





\subsection*{Uppgift 6 -- Bergskedja}


~\\
{\tt 
\begin{tabular}{||l||c|c|l||l||}\hline\hline
& \multicolumn{3}{c||}{\fe{Indata}} & \fe{Utdata} \\
& $n$ & $m$ & Rader & \\ \hline \hline
\fe{Test 1} & 1 & 8 & 11110201 & 5 8 \\ \hline
\fe{Test 2} & 3 & 3 & 130 & 2 8 \\ 
&&& 022& \\
&&& 202& \\ \hline
\fe{Test 3} & 4 & 4 & 1112 & 2 11 \\
&&& 0340 & \\
&&& 3012 & \\
&&& 0312 & \\\hline
\fe{Test 4} & 8 & 7 & 1310222 & 9 55\\
&&& 1142022 & \\
&&& 1403301 & \\
&&& 1330242 & \\
&&& 1112420 & \\
&&& 1414213 & \\
&&& 1240310 & \\
&&& 1021122 & \\\hline
\fe{Test 5} & 8 & 8 & 01111111 & 1 42\\
&&& 13123322 & \\
&&& 24243131 & \\
&&& 03022041 & \\
&&& 30430220 & \\
&&& 04133242 & \\
&&& 31101302 & \\
&&& 03213030 & \\\hline\hline
\end{tabular}
}



\end{document}


