\documentclass[a4paper,12pt,oneside]{amsbook}
\usepackage[T1]{fontenc}
\usepackage{textcomp}
\usepackage{multicol}
%\usepackage{euler}
%\usepackage{garamond}
\usepackage{hyperref}

\usepackage{newcent}
%\fontfamily{fxm}\selectfont
\usepackage{fancyhdr}
\pagestyle{fancy}
% Detta paket skoter grafiken
\usepackage[dvips]{graphicx}
%\usepackage[pdftex]{graphicx}

\usepackage{amsthm}
\usepackage{amsmath}
\usepackage[swedish]{babel}
% For bortkommentering
\usepackage{verbatim}
% For att slippa indentering av Sections och SubSections
\usepackage{indentfirst}

% Har stalls paragrafindentering och avstand mellan stycken och rader mm
\setlength{\parindent}{0pt}
\setlength{\parskip}{6pt}
\setlength{\baselineskip}{12pt}
% Har definieras sidan
\setlength{\voffset}{-15mm}
\setlength{\hoffset}{-5.5mm}

\setlength{\topmargin}{0mm}
\setlength{\headheight}{10mm}
\setlength{\headsep}{8mm}
\setlength{\footskip}{16mm}

\setlength{\evensidemargin}{0mm}
\setlength{\oddsidemargin}{0mm}

\setlength{\marginparsep}{0mm}
\setlength{\marginparwidth}{0mm}

\setlength{\textwidth}{170mm}
\setlength{\textheight}{230mm}
\setlength{\paperwidth}{210mm}
\setlength{\paperheight}{297mm}
\setlength{\headwidth}{170mm}
% Huvud och Fot
%\fancypagestyle{}{
\fancyhf{}\fancyhead[c]{\small{\textit{Programmeringsolympiaden Kvalificering 2020}}}
\fancyfoot[c]{\thepage}
\renewcommand{\headrulewidth}{0.4pt}
\renewcommand{\footrulewidth}{0.4pt}
\newenvironment{lista}
{\begin{itemize}
\setlength{\parindent}{0pt}
\setlength{\itemsep}{6pt}}
{\end{itemize}}
% Uppgifter
\newcounter{probnr}
\newenvironment{tal}{%
\begin{list}
%{\textbf{\arabic{section}.\arabic{probnr}}} {\usecounter{probnr}
{\textbf{\arabic{probnr}}} {\usecounter{probnr}
\setlength{\leftmargin}{0mm}
\setlength{\rightmargin}{0mm}
\setlength{\labelwidth}{-1mm}
\setlength{\labelsep}{1mm}}
\setlength{\itemsep}{6pt}
}{\end{list}}
% Definition av avsnitt: FAKTA, EXEMPEL, PASTAENDE
\newtheorem{fakta}{Fakta}
\newtheorem{exempel}{Exempel}
\newtheorem{problem}{Problem}
%
\newtheoremstyle{test}% NAME
{20pt}      % ABOVESPACE
{10pt}      % BELOWSPACE
{\sffamily} % BODYFONT
{0pt}       % INDENT
{\scshape}  % HEADFONT
{}          % HEADPUNCT
{\newline}  % HEADSPACE
{}          % CUSTOM-HEAD-SPEC

\theoremstyle{test}
\newtheorem{program}{Program}
\newcommand{\sv}[1]{\textsc{#1}}            % Sma versaler
\newcommand{\fe}[1]{\textbf{#1}}            % FET
\newcommand{\ku}[1]{\textit{#1}}            % KURSIV
\newcommand{\cu}[1]{\texttt{#1}}            % Courier
\newcommand{\sk}[1]{\texttt{#1}}            % Courier
\newcommand{\rubrik}[1]{\begin{center}\sf\huge{#1}\normalsize\rm\end{center}}
\begin{document}
%\DeclareGraphicsExtensions{.jpg,.pdf,.mps,.png,.eps}

\problemname{Teleskop}
Hitta på backstory. (kanske att man inte kan stänga av teleskopen när man sätter på den?) Man får välja graden man börjar på och man får starta var som helst. Man måste välja en subsegment



\section*{Indata}
Den första raden innehåller två heltal: $1\leq N \leq 2 \cdot 10^5$ och $0 \leq K \leq 10^9$ - antalet stjärnor och hur mycket bränsle du har.


Den andra raden innehåller N heltal, det $i:$te talet är graden av stjärnan som dyker upp  som visas i den $i:$te minuten.


\section*{Utdata}
Skriv ut ett heltal - det största antalet sjärnonr du kan ta bild på.

\section*{Poängsättning}
Din lösning kommer att testas på en antal testfallsgrupper.
För att få poäng för en grupp så måste du klara alla testfall i gruppen.

\noindent
\begin{tabular}{| l | l | p{12cm} |}
  \hline
  Grupp & Poängvärde & Gränser \\ \hline
  $1$   & $50$       & $N \leq 2000 $\\ \hline
  $2$   & $50$       & $N \leq 2 \cdot 10^5 $  \\ \hline
\end{tabular}

Notera att vissa exempelfall är inte giltiga i alla testfallsgrupper.

\section*{Förklaring av exempelfall 1}


\end{document}