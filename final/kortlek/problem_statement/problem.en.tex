\problemname{Card game}
\noindent
Nicole and Simon are playing a card game consisting of $N$ rounds.
In round $i$, Nicole plays a card with a number $a_i$ written on it.
Simon must respond by playing a card from his hand.
If Simon's card has a value $b_i$, Nicole scores $|a_i - b_i|$ points.
Nicole thus scores more points the farther Simon's card value is from the one Nicole played.

Given exactly which cards Nicole will play and which cards Simon has in his hand from the start, what is the minimum score Nicole can get if Simon plays optimally?
If Simon has $M$ cards, they will always play exactly $M$ or $M - 1$ rounds, i.e., $M = N$ or $M = N + 1$.

\section*{Input}
The first line contains two integers $N$ ($1 \leq N \leq 2 \cdot 10^5$) and $M$ ($N \leq M \leq N+1$), the number of rounds in the game and the number of cards Simon has.

The second line contains $N$ integers $a_1, \dots, a_N$ ($0 \le a_i \le 10^9$), where $a_i$ is the value of the card Nicole plays in round $i$.

The third line contains $M$ integers $b_1, \dots, b_M$ ($0 \le b_i \le 10^9$), the values of Simon's cards.

\section*{Output}
Print the minimum total score Nicole gets if Simon plays optimally.

\section*{Scoring}
Your solution will be tested on a set of test groups, each worth a number of points. Each test group contains
a set of test cases. To get the points for a test group you need to solve all test cases in the test group.

\noindent
\begin{tabular}{| l | l | l |}
  \hline
  \textbf{Group} & \textbf{Points} & \textbf{Constraints} \\ \hline
  $1$   & $15$       & $N \leq 8 $\\ \hline
  $2$   & $25$       & $N \leq 2000 $  \\ \hline
  $3$   & $20$       & $M=N$ \\ \hline
  $4$   & $40$       & No additional constraints. \\ \hline
\end{tabular}

\section*{Explanation of Samples}
In example case 1, it is optimal for Simon to play the card with value 1 in the first round, and the card with value 2 in the second round. Then Nicole gets $|1-1| + |10-2| = 8$ points.

In example case 2, Simon plays the cards with values 2, 5, 1, in that order.

In example case 3, Simon plays the cards with values 4, 6, 3, 1, in that order.
