\problemname{Kortlek}
Nicole och Simon spelar ett kortspel som består av $N$ rundor. I runda $i$ lägger Nicole ut ett kort som har ett tal $a_i$ skrivet på sig. Simon måste då svara med att lägga ut ett kort från sin hand. Om Simons kort har värde $b_i$ så får Nicole $|a_i-b_i|$ poäng. Simon vill alltså lägga ett kort som är så nära det Nicole lade som möjligt.

Givet exakt vilka kort Nicole kommer lägga ut och vilka $M$ kort Simon har på sin hand från början, vad är den minsta poängen Nicole kan få om Simon spelar optimalt? $M$ är alltid lika med $N$ eller $N+1$.

\section*{Indata}
Den första raden innehåller de två heltalen $N$ ($1\leq N \leq 2 \cdot 10^5$) och $M$ ($N\leq M \leq N+1$).

Den andra raden innehåller $N$ heltal, där det $i$:te talet $a_i$ ($0\le a_i \le 10^9$) är värdet på kortet Nicole lägger ut i runda $i$.

Den tredje raden innehåller $M$ heltal, där det $i$:te talet $b_i$ ($0\le b_i \le 10^9$) är värdet av det $i:$te kortet Simon har på sin hand.

\section*{Utdata}
Skriv ut ett heltal -- den minsta totala poängen Nicole får om Simon spelar optimalt.

\section*{Poängsättning}
Din lösning kommer att testas på en mängd testfallsgrupper.
För att få poäng för en grupp så måste du klara alla testfall i gruppen.

\noindent
\begin{tabular}{| l | l | p{12cm} |}
  \hline
  Grupp & Poängvärde & Gränser \\ \hline
  $1$   & $15$       & $N \leq 8 $\\ \hline
  $2$   & $20$       & $N \leq 2000 $  \\ \hline
  $3$   & $25$       & $M=N$ \\ \hline
  $4$   & $40$       & Inga ytterligare begränsningar \\ \hline
\end{tabular}

\section*{Förklaring av exempelfall}
I exempelfall 1 är det optimalt för Simon att i första rundan lägga ut kortet med värde 1, och i andra rundan lägga ut kortet med värde 2. Då får Nicole $|1-1| + |10-2|=8$ poäng.

I exempelfall 2 spelar Simon ut korten av värde 2, 5, 1, i den ordningen.

I exempelfall 3 spelar Simon ut korten av värde 4, 6, 3, 1, i den ordningen.
