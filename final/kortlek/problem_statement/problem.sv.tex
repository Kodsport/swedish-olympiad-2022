\problemname{Kortlek}
I ett kortlek som består av $N$ rundor visas ett tal per runda och varje spelare ska lägga ett kort. Om värdet som visas är $y$ och spelaren använder ett kort med värde $x$ får spelaren $|x-y|$ som skuld. Givet att du har $M$ kort, vad är den minimala skulden du kan få? $M$ är alltid lika med antingen $N$ eller $N+1$



\section*{Indata}
Den första raden innehåller två heltal: $1\leq N \leq 2 \cdot 10^5$ och $N\leq M \leq N+1$

Den andra raden innehåller N heltal, det $i:$te talet är värdet som visas i den $i:$te rundan.

Den tredje raden innehåller M heltat, det $i:$te talet är värdet av det $i:$te kortet du har.

Alla tal är mellan $0$ och $10^9$ inklusive i både $0$ och $10^9$

\section*{Utdata}
Skriv ut ett heltal - den minsta totala skulden du kan få i slutet av spelet.

\section*{Poängsättning}
Din lösning kommer att testas på en antal testfallsgrupper.
För att få poäng för en grupp så måste du klara alla testfall i gruppen.

\noindent
\begin{tabular}{| l | l | p{12cm} |}
  \hline
  Grupp & Poängvärde & Gränser \\ \hline
  $1$   & $15$       & $N \leq 8 $\\ \hline
  $2$   & $20$       & $N \leq 2000 $  \\ \hline
  $3$   & $25$       & $M=N$ \\ \hline
  $4$   & $40$       & Inga ytterligare begränsningar \\ \hline
\end{tabular}

Notera att vissa exempelfall är inte giltiga i alla testfallsgrupper.

\section*{Förklaring av exempelfall 1}
