\problemname{Växter}
Längs en $N$ meter lång väg finns det $N$ växter. Växten $i$ är av arten $A_i$ och är exakt en meter ifrån växten $A_{i-1}$. En växtforskare vill gå längs vägen för att samla växter. Då han har tillbringat hela sin tid i sitt labb så har han inte bra kondition,
och därför frågar han dig $Q$ frågor på formen: om jag vill samla alla växter av arterna som är mellan $0$ och $Q_i$ hur många växter behöver jag gå förbi som minst? 
Han kan börja var som helst och sluta var som helst och han kan gå förbi en växt utan att samla den.

\section*{Indata}
Den första raden innehåller de två heltalen $N$ ($1\leq N \leq 2 \cdot 10^5$) och $Q$ ($1 \leq Q \leq 2 \cdot 10^5$).

Den andra raden innehåller $N$ heltal, där det $i$:te talet $A_i$ ($0\le A_i \le 10^9$) är arten av växten $i$.

Därefter följer $Q$ rader, den $i$:te raden innehåller ett tal$Q_i$ ($0 \leq Q_i \leq 10^9$) som representerar den $i$:te frågan växtforskaren ställer till dig.

\section*{Utdata}
Skriv ut $Q$ rader där den $i$:te raden innehåller svaret på den $i$:te frågan. Om det är omöjligt att samla växter mellan $0$ och $Q_i$, skriv ut $-1$.

\section*{Poängsättning}
Din lösning kommer att testas på en mängd testfallsgrupper.
För att få poäng för en grupp så måste du klara alla testfall i gruppen.

\noindent
\begin{tabular}{| l | l | p{12cm} |}
  \hline
  Grupp & Poängvärde & Gränser \\ \hline
  $1$   & $5$       & $N, Q, A_i, Q_i \leq 200 $\\ \hline
  $2$   & $7$       & $N \leq 200 $\\ \hline
  $3$   & $9$       & $Q=1, Q_i=1$\\ \hline
  $4$   & $12$       & Alla tal $A_i$ är olika. \\ \hline
  $5$   & $16$       & $N \leq 2000 $  \\ \hline
  $6$   & $18$       & $Q = 1$ \\ \hline
  $7$   & $33$       & Inga ytterligare begränsningar. \\ \hline
\end{tabular}

