\problemname{Växter}
Längs en lång väg finns det $N$ växter, växten $i$ är av arten $A_i$. En växt forskare vill gå längs vägen för att samla växter. Då han har tillbringat hela sin tid i sin labb så har han inte bra kondition,
därför frågar han dig $Q$ frågor $i$ formen: om jag vill samla alla växter av arterna som är mellan $0$ och $Q_i$ hur mycket behöver jag gå som minst?
Han kan börja var som helst och sluta var som helst och han kan gå förbi en växt utan att samla den
\section*{Indata}
Den första raden innehåller de två heltalen $N$ ($1\leq N \leq 2 \cdot 10^5$) och $Q$ ($1 \leq Q \leq 2 \cdot 10^5$).

Den andra raden innehåller $N$ heltal, där det $i$:te talet $A_i$ ($0\le A_i \le 10^9$) är arten av växten $i$.

Därefter följer $Q$ rader, den $i$:te raden innehåller ett tal$Q_i$ ($0 \leq Q_i \leq 10^9$) som representerar den $i$:te frågan växtforskaren ställer till dig.

\section*{Utdata}
Skriv ut $Q$ rader som -- i den $i$:te raden ska innehålla svaret av den $i$:te frågan. Om det är omöjligt att samla växter mellan $0$ och $Q_i$  skriv ut $-1$;

\section*{Poängsättning}
Din lösning kommer att testas på en mängd testfallsgrupper.
För att få poäng för en grupp så måste du klara alla testfall i gruppen.

\noindent
\begin{tabular}{| l | l | p{12cm} |}
  \hline
  Grupp & Poängvärde & Gränser \\ \hline
  $1$   & $5$       & $N \leq 300 Q \leq 300 $\\ \hline
  $2$   & $7$       & $N \leq 300 $\\ \hline
  $3$   & $15$       & $N \leq 2000 $  \\ \hline
  $4$   & $30$       &  För alla frågor är det garanterat att det finns en ökande subsekvens som ger optimalt svar \\ \hline
  $5$   & $43$       & Inga ytterligare begränsningar \\ \hline
\end{tabular}

\section*{Förklaring av exempelfall}
