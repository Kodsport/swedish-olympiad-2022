\problemname{Brinnande träd}

En skogsbrand har brutit ut i ett naturreservat. I området finns det en sällsynt art av träd
som inte finns någon annanstans i världen, och så många som möjligt av träden måste räddas.

Totalt finns det $N$ sällsynta träd. Det $i$:te trädet kommer slukas av elden om $D_i$ sekunder,
och det tar $A_i$ sekunder att rädda det. Du kan bara rädda ett träd i taget. Om ett träd nås
av elden medan du håller på att rädda det så brinner det upp. Om du däremot blir klar med att rädda
trädet exakt samma sekund som elden kommer, så lyckas räddningen (se exempel $2$).

Tiden det tar för att rädda ett träd beror bland annat på avståndet till trädet. Därför uppfyller
talen $A_i, D_i$ följande triangelolikhetsliknande relation: Om $D_j > D_i$, så gäller det att 

$$A_j - A_i \leq D_j - D_i$$

\section*{Indata}

Den första raden innehåller Ett heltal $N$ ($1 \leq N \leq 4 \cdot 10^5$).

Den andra raden innehåller $N$ heltal $D_i$ ($1 \leq D_i \leq 10^9$).

Den tredje raden innehåller $N$ heltal $A_i$ ($1 \leq A_i \leq 10^9$).

\section*{Utdata}
Skriv ut ett heltal, det maximala antalet träd som går att rädda.

\section*{Poängsättning}
Din lösning kommer att testas på en mängd testfallsgrupper.
För att få poäng för en grupp så måste du klara alla testfall i gruppen.

\noindent
\begin{tabular}{| l | l | p{12cm} |}
  \hline
  Grupp & Poängvärde & Gränser \\ \hline
  $1$   & $14$       & $A_1 = A_2 = \cdots = A_N$\\ \hline
  $2$   & $16$       & $D_1 = D_2 = \cdots = D_N$  \\ \hline
  $3$   & $11$       & $N, A_i, D_i \leq 100$ \\ \hline
  $3$   & $23$       & $N \leq 2000$ \\ \hline
  $4$   & $36$       & Inga ytterligare begränsningar \\ \hline
\end{tabular}